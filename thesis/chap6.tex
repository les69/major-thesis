\chapter{Conclusion}
\label{chap5}
Throughout in our document we proposed some approaches designed
to solve some of the trending problems when creating House Security systems
with IoT. \newline
Integration among different devices is a hard subject being topic of research
for some years, with different objectives and solutions. Our solution does not aim
to be the perfect solution for all the integration problems but rather
a better approach for the intercommunication between heterogeneous ecosystems.
Furthermore we adopted a new approach, microservices, with a clear vision for
upgradability and sustainability in a future. Microservices does not only
solve many of our problems, but helps us building a system with some key
properties such as upgradability, resiliency and flexibility. Moreover
this approach fit very well in Calvin's architecture, making it easier to develop
and maintain. There is of course a price to be paid for the microservice architecture,
resulting from the overhead introduced by the chain of services: a higher execution time.
As we have seen in the many different test, this overhead doesn't impact highly on the
overall performance, remaining by far under an acceptable threshold. By that we mean
the overall performance impact of microservices is not high enough to introduce
any relevant delay when reading or writing values from sensors in different
ecosystems.\newline
One of the many aspects which has to be considered in a house security scenario is the
number of false positives for intrusion detection. There are already a lot of different
techniques to detect intruders and to reduce false positives for their use cases.
However we decided to try a different path using speaker recognition, more precisely
speaker identification, for revealing an intruder (possibly active only when there's no one in the house).
Calvin is a perfect framework for the task, given its distributed architecture exploiting the
key aspects of different characteristics. Having many actors with different tasks
running on different types machines enables us to have high performances
while accessing sensors from low hardware devices.\newline
We built a speaker recognition system with Calvin using a distributed architecture,
getting the input from one machine and processing it on a different one. All that using
actors to model the flow of the execution.
The overall system proved its validity by validating many of our assumptions and
invalidating some others. The basic of recognition worked very well, and through
our tests we learned the right set up for the different audio recording duration length (model and test)
in order to achieve the best result. The results outlined some key aspects we have to consider
before building such a system: the input quality, the background noise and the silences between the speech.
All these aspects can influence the score, though they can be solved easily. Although many
tests succeeded, many others failed, showing the limits of the speaker recognition system. Anyway those
tests which failed were not impacting too heavily on the system overall feasibility, they were rather
limit cases were it could have worker or not.\newline
Finally we outlined the possibility for our system to fully recognize people with short
audio models and for it being also able to learn and verify dog's by their barking. Furthermore
the microservice system itself proved to be robust enough to handle even more than the typical
traffic it should in a real world scenario.


\subsection{Future work}

Although our system proved its effectiveness it's still a Proof-of-Concept,
far from a final deployment. Even though the techniques we used to integrate with
different ecosystems use working solutions, some of them use solutions
which could be improved in a future. For example, running a web service for integration
with the ecosystem is not the best solution, but it's an actual limit imposed by Apple.
It is very likely that in a future they will allow access to Homekit also from other
operative systems like tvOS or Mac OS X. The approach is working, and moving it to a better
platform won't be hard as starting from scratch.\newline
The libraries we have used for speaker recognition uses a custom version of GMM-UBM algorithm,
which works smoothly but there are more recent and performant state-of-the-art algorithms,
as shown in Chapter \ref{chap2}. In the future these algorithms can be tested under the same
situations and see the benefit introduced by them and if they can pass the test were
our failed.
