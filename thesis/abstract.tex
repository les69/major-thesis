% $Log: abstract.tex,v $
% Revision 1.1  93/05/14  14:56:25  starflt
% Initial revision
%
% Revision 1.1  90/05/04  10:41:01  lwvanels
% Initial revision
%
%
%% The text of your abstract and nothing else (other than comments) goes here.
%% It will be single-spaced and the rest of the text that is supposed to go on
%% the abstract page will be generated by the abstractpage environment.  This
%% file should be x\input (not \include 'd) from cover.tex.
\begin{abstract}

The recent increase in power of computation combined with the decrease of power
consumption of devices led to an explosion of smart devices for all the possible
different purposes, called the \textit{Internet of Things} (IoT).\\
The IoT changed many daily habits automating the usual tasks in a more efficient
way, improving costs for electrical appliances usage. The latter is just
one of the many examples of benefits introduced by common smart devices.
Some of these just helped people to improve their daily life, whereas other
aimed for a bigger purpose, for example \textit{securing the house}.
Many solutions have been proposed in the recent years, more and more sophisticated
as the computational power of smart devices allowed it to grow. However,
with the use of different technologies between ecosystems, intercommunication
became more and more complex. Different systems use different technologies, and
the solution for the problem is not trivial. All the approaches proposed
in the past do all have their benefits, though in our scenario we moved
towards the \textit{microservice architecture}. Microservices are a possible solution
for the problem, and their benefits are explained later on in the document.\\
Security systems are also victims of \textit{false positives}, when a system
detects an intruder when there is no intrusion. There are already many techniques
applied for recognizing intruders, some are invasive and some other less invasive. In
this document we focused on \textit{speaker recognition} to detect eventual intruders, as a more advanced
measure for intrusion detection.\\
During the course of the document we combined all our approaches with a cloud oriented
framework for IoT, namely \textit{Calvin}.

\end{abstract}
