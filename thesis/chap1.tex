%% This is an example first chapter.  You should put chapter/appendix that you
%% write into a separate file, and add a line \include{yourfilename} to
%% main.tex, where `yourfilename.tex' is the name of the chapter/appendix file.
%% You can process specific files by typing their names in at the
%% \files=
%% prompt when you run the file main.tex through LaTeX.
\chapter{Introduction}

IoT surveillance systems has been proved to be effective during the time,
catching the burglar committing the crime and helping the authorities to
catch him[1]. Nonetheless these systems are not yet perfect, and during
this paper we will investigate on some optimizations with respect to
the \textit{integration} with other ecosystems and the \textit{reduction}
of false positives.\\
Here follows the main structure of the document:\\

Chapter two introduces the technologies used in our scenario, with a brief
introduction on the techniques adopted to face the various problems.

Chapter three illustrates the main architecture of the optimizations proposed
during the document and the motivation behind them.

Chapter four shows the implementation of our use case with the optimizations
we have described formerly.

Chapter five is an extensive analysis of the results obtained during the whole
analysis of the scenario.

\section{Motivations}

The idea behind our research in this field is to improve the existing solution
with an innovative and different approach. Nowdays there are many standalone
solutions for IoT surveillance, which acts separately from all the other
components in the house. We address this isolation, trying to create
a system capable of interconnecting with other existing components in the house.
There are currently 4 billion of connected devices [2] as 2016, and the forecasts
says they will be 13.5b in 2020. This means a growth in the heterogenity of companies
and products, which makes a must the interopability between devices from different
producers.\\
Most of the common Smart devices are built to work appositely with their own application,
without any external support which leads to a loose coupling between devices.

%%[TODO] add stuff here, lacking of inspiration now

\section{Description of scenario}\label{ch1:opts}

During this document we will refer to a use case scenario related
to house security. In this scenario we will have a small house with different
floors, each floor having a leap motion sensor to detect any movement and a camera
recording.
\subsection{M2M: Machine to Machine}

Most of common smart sensors, in order to be \textit{smart} they need
to provide a form of connectivity: let it be BLE (Bluetooth Low Energy),
Wi-Fi or R-FID. \textit{M2M} is treanding with the raising of IoT,
requiring a higher number of devices to be interconnected without
any human interaction. Different devices means different protocols,
which introduces difficulties in the communications between each other.
The typical approach is to define a set of translators which
are able to deal with both sides, also called \textit{Gateways}.
Gateways are high-level objects that knows the various protocols
needed to communicate and provide these knowledge as a service to
whom has the necessity to access the functionalities provided by the
device. \\
In our architecture each gateway is a device exposing a service using
a \textbf{RESTful} architecture, for a better simplicity of use.

\subsection{Microservices and IoT}
\label{sub:microservicesiot}

The Microservice architecture is an innovative modelling pattern that aims
to solve a well defined class of problems: scaling.
The idea behind microservices is to split the architecture on different machines
usually communicating through a RESTful interface.
This pattern is similar to the \textit{microkernel} architecture for Operative Systems,
where the kernel containes the most vital functions and each functionality is a Plug-In
(or Driver Module) that can be added externally.
Each service holds a functionality isolated in it's context, which can be deployed
at runtime without any interruption of the service. In microservices the
equivalent of the kernel is the API interface that exposes all the functionalities
to the external world.
With the raising of the need of interopability, \textit{microservices}
seems to hold the key for this problem.
Vendors have published their protocols, and have exposed API’s to their various hubs.
A MicroService can serve as an adapter between various protocols. It can be lightweight and disposable,
both desirable traits in a rapidly evolving environment.[3]



\subsection{Block Exponent}

In a unoptimized sequence of additions, the sequence of operations is as
follows for each pair of numbers ($m_1$,$e_1$) and ($m_2$,$e_2$).
\begin{enumerate}
  \item Compare $e_1$ and $e_2$.
  \item Shift the mantissa associated with the smaller exponent $|e_1-e_2|$
        places to the right.
  \item Add $m_1$ and $m_2$.
  \item Find the first one in the resulting mantissa.
  \item Shift the resulting mantissa so that normalized
  \item Adjust the exponent accordingly.
\end{enumerate}

Out of 6 steps, only one is the actual addition, and the rest are involved
in aligning the mantissas prior to the add, and then normalizing the result
afterward.  In the block exponent optimization, the largest mantissa is
found to start with, and all the mantissa's shifted before any additions
take place.  Once the mantissas have been shifted, the additions can take
place one after another\footnote{This requires that for n consecutive
additions, there are $\log_{2}n$ high guard bits to prevent overflow.  In
the $\mu$FPU, there are 3 guard bits, making up to 8 consecutive additions
possible.}.  An example of the Block Exponent optimization on the expression
X = A + B + C is given in figure~\ref{opt:be}.

% This is an example of how you would use tgrind to include an example
% of source code; it is commented out in this template since the code
% example file does not exist.  To use it, you need to remove the '%' on the
% beginning of the line, and insert your own information in the call.
%
%\tgrind[htbp]{code/be.s.tex}{Block Exponent}{opt:be}

\section{Integer optimizations}

As well as the floating point optimizations described above, there are
also integer optimizations that can be used in the $\mu$FPU.  In concert
with the floating point optimizations, these can provide a significant
speedup.

\subsection{Conversion to fixed point}

Integer operations are much faster than floating point operations; if it is
possible to replace floating point operations with fixed point operations,
this would provide a significant increase in speed.

This conversion can either take place automatically or or based on a
specific request from the programmer.  To do this automatically, the
compiler must either be very smart, or play fast and loose with the accuracy
and precision of the programmer's variables.  To be ``smart'', the computer
must track the ranges of all the floating point variables through the
program, and then see if there are any potential candidates for conversion
to floating point.  This technique is discussed further in
section~\ref{range-tracking}, where it was implemented.

The other way to do this is to rely on specific hints from the programmer
that a certain value will only assume a specific range, and that only a
specific precision is desired.  This is somewhat more taxing on the
programmer, in that he has to know the ranges that his values will take at
declaration time (something normally abstracted away), but it does provide
the opportunity for fine-tuning already working code.

Potential applications of this would be simulation programs, where the
variable represents some physical quantity; the constraints of the physical
system may provide bounds on the range the variable can take.
\subsection{Small Constant Multiplications}

One other class of optimizations that can be done is to replace
multiplications by small integer constants into some combination of
additions and shifts.  Addition and shifting can be significantly faster
than multiplication.  This is done by using some combination of
\begin{eqnarray*}
a_i & = & a_j + a_k \\
a_i & = & 2a_j + a_k \\
a_i & = & 4a_j + a_k \\
a_i & = & 8a_j + a_k \\
a_i & = & a_j - a_k \\
a_i & = & a_j \ll m \mbox{shift}
\end{eqnarray*}
instead of the multiplication.  For example, to multiply $s$ by 10 and store
the result in $r$, you could use:
\begin{eqnarray*}
r & = & 4s + s\\
r & = & r + r
\end{eqnarray*}
Or by 59:
\begin{eqnarray*}
t & = & 2s + s \\
r & = & 2t + s \\
r & = & 8r + t
\end{eqnarray*}
Similar combinations can be found for almost all of the smaller
integers\footnote{This optimization is only an ``optimization'', of course,
when the amount of time spent on the shifts and adds is less than the time
that would be spent doing the multiplication.  Since the time costs of these
operations are known to the compiler in order for it to do scheduling, it is
easy for the compiler to determine when this optimization is worth using.}.
\cite{magenheimer:precision}

\section{Other optimizations}

\subsection{Low-level parallelism}

The current trend is towards duplicating hardware at the lowest level to
provide parallelism\footnote{This can been seen in the i860; floating point
additions and multiplications can proceed at the same time, and the RISC
core be moving data in and out of the floating point registers and providing
flow control at the same time the floating point units are active. \cite{byte:i860}}

Conceptually, it is easy to take advantage to low-level parallelism in the
instruction stream by simply adding more functional units to the $\mu$FPU,
widening the instruction word to control them, and then scheduling as many
operations to take place at one time as possible.

However, simply adding more functional units can only be done so many times;
there is only a limited amount of parallelism directly available in the
instruction stream, and without it, much of the extra resources will go to
waste.  One process used to make more instructions potentially schedulable
at any given time is ``trace scheduling''.  This technique originated in the
Bulldog compiler for the original VLIW machine, the ELI-512.
\cite{ellis:bulldog,colwell:vliw}  In trace scheduling, code can be
scheduled through many basic blocks at one time, following a single
potential ``trace'' of program execution.  In this way, instructions that
{\em might\/} be executed depending on a conditional branch further down in
the instruction stream are scheduled, allowing an increase in the potential
parallelism.  To account for the cases where the expected branch wasn't
taken, correction code is inserted after the branches to undo the effects of
any prematurely executed instructions.

\subsection{Pipeline optimizations}

In addition to having operations going on in parallel across functional
units, it is also typical to have several operations in various stages of
completion in each unit.  This pipelining allows the throughput of the
functional units to be increased, with no increase in latency.

There are several ways pipelined operations can be optimized.  On the
hardware side, support can be added to allow data to be recirculated back
into the beginning of the pipeline from the end, saving a trip through the
registers.  On the software side, the compiler can utilize several tricks to
try to fill up as many of the pipeline delay slots as possible, as
seendescribed by Gibbons. \cite{gib86}
