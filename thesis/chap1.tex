%% This is an example first chapter.  You should put chapter/appendix that you
%% write into a separate file, and add a line \include{yourfilename} to
%% main.tex, where `yourfilename.tex' is the name of the chapter/appendix file.
%% You can process specific files by typing their names in at the
%% \files=
%% prompt when you run the file main.tex through LaTeX.
\chapter{Introduction}
\label{chap1}


Internet of Things(IoT) based surveillance systems have been proved to be effective during the time,
catching burglars committing the crime and helping the authorities to
catch him\cite{fbrobbery}. Nonetheless these systems are not yet perfect, and during
this document we will investigate on some optimizations with respect to
the \textit{integration} with other ecosystems and the \textit{reduction}
of false positives.\\
Here follows the main structure of the thesis:\\

Chapter \ref{chap2} introduces the technologies used in our scenario, with a brief
introduction on the techniques adopted to face the various problems.

Chapter \ref{chap3} illustrates the main architecture proposed
during the document and its implementation.

Chapter \ref{chap4} shows the validation of the architecture proposed in the former
chapter, showing the real benefits of the approach.\\
Conclusions can be found in Chapter \ref{chap5}


\section{Motivation}

In this document we illustrate the common problems that arise
when introducing a new IoT system in an existing environment.
The first motivating problem taken in consideration is the \textit{integration} of a new
system in a house where there's already a different coexisting ecosystem.
The idea behind our research in this field is to improve the existing solutions
with an innovative and different approach. Nowadays there are many standalone
solutions for IoT surveillance, which acts separately from all the other
components in the house. We address this isolation, trying to create
a system capable of interconnecting with other existing components in the house.
There are currently 4 billion of connected devices \cite{gartner} as of 2016, and the forecasts
say there will be 13.5 billion in 2020. This means a growth in the heterogeneity of companies
and products, which makes a must the interoperability between devices from different
producers.\\
Most of the common Smart devices are built to work specifically with their own application,
without any external support which leads to a loose coupling between devices. This problem
arises when introducing new \textit{systems} in a house where there are is already a different
ecosystem. The inability of a user to switch or adopt a new technology is also called \textit{vendor lock-in},
a common commercial practice to retain users. However research is moving forward
this direction investigating in new innovative and efficient methods to solve this
problem, and there are already some approaches which will be discussed in the next chapter.
\textit{False positives} is another key issue in our investigation on the improvement of a
house monitoring scenario. The reason behind this choice lies in the connection with the "idea"
of a house with an alarm connection to a third party security agency. We need to have a real
alarm when something is happening, reducing the probability of false alarms and not wasting time
calling the police or alerting some vigilance agent.


\section{The Scenario}

In our document we will often refer to an imaginary scenario of a typical house
of an average person. It is important to denote this detail since most of our
solutions are designed for a peculiar situation, and they may not be as
efficient or innovative in a different scenario.\\
The main idea is to have a surveillance system in a typical \textit{American style}
house, with different floors and rooms, each of them equipped with different
and various smart devices. Having a clear idea of the scenario helps to focus more
on the identification of issues while designing the systems rather than facing
and solving problems at the last moment due to some unpredictable variables of a
too broad scenario.

\section{Calvin}

\textit{Calvin}\cite{calvin-article} is an open-source and cloud oriented framework
for IoT applications. Calvin is built to scale well, offering the opportunity
to run it on both small devices and more powerful machines in order to use the
full power of cloud computing. Its distributed perspective follows the principle
of \textit{run it where it's most needed}, exploiting the high interoperability with
hardware of small devices combined to the more powerful execution of server machines.
Calvin aims to be easy to develop, using an actor model architecture where actors are
simple but reactive entities interacting with each other.

\section{Microservices and IoT}

The Microservice architecture is an innovative modeling pattern that aims
to solve a well defined class of problems: scaling.
The idea behind microservices is to split the architecture on different machines
usually communicating through a RESTful interface.
In microservices the
functionalities are exposed through an API interface to the external world using the
\textit{gateway pattern}. Microservices share many key aspects with the \textit{microkernel}
architecture pattern, offering a better reliability, resiliency, portability, extensibility,
flexibility and easier and faster environment for development and testing.
With the advent of the need of interoperability, \textit{microservices}
seems to hold the key for this problem.
A MicroService can serve as an adapter between various protocols. It can be lightweight and disposable,
both desirable traits in a rapidly evolving environment.\cite{microiot}

\subsection{M2M: Machine to Machine}

Most of common smart sensors, in order to be \textit{smart}, they need
to provide a form of connectivity: let it be BLE (Bluetooth Low Energy),
Wi-Fi or RFID. \textit{M2M} is trending with the raising of IoT,
requiring a higher number of devices to be interconnected without
any human interaction. Different devices means different protocols,
which introduces difficulties in the communications between each other.
The typical approach is to define a set of translators which
are able to deal with both sides, also called \textit{Gateways}.
Gateways are high-level objects that knows the various protocols
needed to communicate and provide these knowledge as a service to
whom has the necessity to access the functionalities provided by the
device. \\
In our architecture each gateway is a device exposing a service using
a \textit{RESTful} architecture, for a higher simplicity of use.

\subsection{RESTful interfaces}
The communication between different devices is a common and dated
problem in \textit{Computer Science}. Many approaches has been introduced
during time, all of them with their relative pro and cons:
\pagebreak
\begin{itemize}
  \item \textbf{Remote Procedure Calls} used to execute procedures on a remote machine,
  with the advent of \textit{CORBA} it was also possible to do that on machine
  with a different technology stack on it.
  \item The \textbf{Message Oriented Middleware} used queues to exchange messages to
  different platforms using different technologies. This subsequently led to
  the development of the common \textit{ESB} technology.
  \item \textbf{SOAP} a high-level protocol for communication between web services
  over many popular application layer protocols such as HTTP or SMTP.
\end{itemize}

All of them are good candidates for the construction of a distributed system,
but it's better to adopt a single protocol for communication instead of many to
avoid over complicating the problem.
We choose to use the \textit{RESTful} architecture, which
better suits our scenario. Our decision complies with the current trend in the
implementation of microservices, where REST is the typical choice for exposing
functionalities. Its simplicity made it very popular and widely adopted, making
REST one of the "standards" for communication between services on the web, mainly
due to its capability to work over \textit{HTTP} without any additional infrastructure.



\section{False positives}

One of the most common problems that occurs when an alarm
is armed are false positives. In most of the cases it can be just a pet
triggering the alarm, some other times may be just yourself when you forget to
deactivate the alarm.\\
Although this problem may look relatively unimportant, it assumes importance if the
alarmed is linked to a surveillance company which is supposed to look after your house.
That's why during this thesis we will take an effort to address this common issue with different
approaches.\\
The false positive detection is calculated as a probability of an event happening, influenced
by many factors. Each factor is one of the approaches adopted below, which will have a positive or
negative influence on this factor, determining if with a certain precision a detection can be classified
as a \textit{false positive}.
 \pagebreak
\subsection{Approaches}
Here's a list of the proposed approaches taken in consideration during this document

\begin{description}
  \item[Face detection] Detecting a face when the alarm is triggered is a useful way of
  discriminating between a false positive and a real intrusion. However this approach has some limitations
  given by the position of the camera and the speed of the shutter which impacts the image quality.
  Its importance differs if used in combination with other different \textit{false positives} reduction
  systems.
  \item[Face recognition] Face recognition is just an improvement of the above technique, which would allow
  us to recognize a familiar face in the system and reduce significantly the probability of a false alarm.
  This approach however is affected by the same problems from the former one.
  \item[Speaker recognition](text independent) This approach involves machine learning to learn and recognize familiar voices
  during an intrusion. Knowing with a certain precision that the voice belongs to someone familiar will
  drastically drop the probability of a possible false detection. Furthermore this approach does not suffer
  from the same problems as the former two approaches.
  (text dependent) People's biometric signature also includes the way they do talk. It is
  actually possible to identify someone by the words they say.
\end{description}
