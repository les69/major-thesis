%% This is an example first chapter.  You should put chapter/appendix that you
%% write into a separate file, and add a line \include{yourfilename} to
%% main.tex, where `yourfilename.tex' is the name of the chapter/appendix file.
%% You can process specific files by typing their names in at the
%% \files=
%% prompt when you run the file main.tex through LaTeX.
\chapter{Introduction}

IoT surveillance systems has been proved to be effective during the time,
catching the burglar committing the crime and helping the authorities to
catch him[1]. Nonetheless these systems are not yet perfect, and during
this paper we will investigate on some optimizations with respect to
the \textit{integration} with other ecosystems and the \textit{reduction}
of false positives.\\
Here follows the main structure of the document:\\

Chapter two introduces the technologies used in our scenario, with a brief
introduction on the techniques adopted to face the various problems.

Chapter three illustrates the main architecture of the optimizations proposed
during the document and the motivation behind them.

Chapter four shows the implementation of our use case with the optimizations
we have described formerly.

Chapter five is an extensive analysis of the results obtained during the whole
analysis of the scenario.

\section{Motivations}

The idea behind our research in this field is to improve the existing solutions
with an innovative and different approach. Nowdays there are many standalone
solutions for IoT surveillance, which acts separately from all the other
components in the house. We address this isolation, trying to create
a system capable of interconnecting with other existing components in the house.
There are currently 4 billion of connected devices [2] as 2016, and the forecasts
says they will be 13.5b in 2020. This means a growth in the heterogenity of companies
and products, which makes a must the interopability between devices from different
producers.\\
Most of the common Smart devices are built to work specifically with their own application,
without any external support which leads to a loose coupling between devices.

%%[TODO] add stuff here, lacking of inspiration now

\section{Description of scenario}\label{ch1:opts}

During this document we will refer to a use case scenario related
to house security. In this scenario we will have a small house with different
floors, each floor having a leap motion sensor to detect any movement and a camera
recording.
\subsection{M2M: Machine to Machine}

Most of common smart sensors, in order to be \textit{smart} they need
to provide a form of connectivity: let it be BLE (Bluetooth Low Energy),
Wi-Fi or R-FID. \textit{M2M} is treanding with the raising of IoT,
requiring a higher number of devices to be interconnected without
any human interaction. Different devices means different protocols,
which introduces difficulties in the communications between each other.
The typical approach is to define a set of translators which
are able to deal with both sides, also called \textit{Gateways}.
Gateways are high-level objects that knows the various protocols
needed to communicate and provide these knowledge as a service to
whom has the necessity to access the functionalities provided by the
device. \\
In our architecture each gateway is a device exposing a service using
a \textbf{RESTful} architecture, for a better simplicity of use.

\section{Microservices and IoT}
\label{sub:microservicesiot}

The Microservice architecture is an innovative modelling pattern that aims
to solve a well defined class of problems: scaling.
The idea behind microservices is to split the architecture on different machines
usually communicating through a RESTful interface.
This pattern is similar to the \textit{microkernel} architecture for Operative Systems,
where the kernel containes the most vital functions and each functionality is a Plug-In
(or Driver Module) that can be added externally.
Each service holds a functionality isolated in it's context, which can be deployed
at runtime without any interruption of the service. In microservices the
equivalent of the kernel is the API interface that exposes all the functionalities
to the external world.
With the raising of the need of interopability, \textit{microservices}
seems to hold the key for this problem.
Vendors have published their protocols, and have exposed API’s to their various hubs.
A MicroService can serve as an adapter between various protocols. It can be lightweight and disposable,
both desirable traits in a rapidly evolving environment.[3]



\section{The power of Social Networks}

Social networks plays a fundamental role in people's life nowdays,
keeping in touch people from different countries or sharing their life
moments with everyone. On average American's spend 4.7 hours browsing on their
social profile, usually around 17 times per day [4].\\
However social networks opened the door for some important issues for people in their
daily/working life
\begin{description}
    \item[Productivity] The most common problem with social networks is of course
    the reduction of productivity/attention that affects people impacting negatively on their life.
    \item[Scams] Scam are very common on social networks, people using fake profiles to perpetrate
    illegal actions such as blackmailing or phishing.
    \item[Spying] This problem is the less recognized but most dangerous vulnerability introduced by social networks.
    Basically, if you share you're entire life with your friends, there may be someone else intrested in your activity: thieves.
    Burglars checks people's social activity with fake accounts or looking at who has checked in in airports to select the houses
    to break into. This is the case we will work in the following pages of the document.
\end{description}

Altough their vulnerabilities, social networks has the power to connect thousands or millions of people togheter,
allowing users to spread news at unthinkable speed compared to traditional methods. Let's consider
Belgium's police, who uses twitter to share the terrorists video hoping to reach someone who
can recognize them.[5] \\
Furthermore this has been proved to be working, when a woman recorded a burglar
who intruded in her house, after she shared her position, and tracked him
posting his image on \textit{Facebook}.[6] This episode is a clear example
of how social networks can be exploited to use their power, and we
will apply this concept in our scenario.

\section{False positives}

\subsection{Approaches}
